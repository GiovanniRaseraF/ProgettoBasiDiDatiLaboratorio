\documentclass[legalpaper]{article}
\usepackage[legalpaper, margin=1in]{geometry}
\usepackage[T1]{fontenc}
\usepackage[utf8]{inputenc}
\usepackage[italian]{babel}
\begin{document}

\section{Progettazione Fisica}
\subsection{Hardware e Tecnologie Utilizzate}
Per testare e condividre il database a tutti i membri abbiamo creato un server utilizzando un \textbf{Raspberry Pi 3} con all'interno \textbf{Raspbian}.
\begin{itemize}
\item CPU: Arm Cortex-A7
\item RAM: 1GB
\end{itemize}
Dopo aver installato il sistema operativo abbiamo potuto interfacciarci tramite \textbf{openssh-server}.\\
A questo punto abbiamo installato \textbf{postgresql-11} seguendo le istruzioni riportate dalla guida ufficiale.
\begin{itemize}
\item Installazione: https://www.postgresql.org/download/linux/ubuntu/
\end{itemize}
Abbiamo quindi a disposizione la console di postgres che ci permette di eseguire le query simile all'interazione \textbf{tty}; abbiamo quindi deciso di permettere a postgres di accettare le connessioni
da parte di client \textbf{pgAdmin4}.\newline
Per farlo c'è bisogno di modificare i file di conficurazioni iniziale di postgresql; nel nostro caso si trovano in \\ \textbf{/etc/postgresql/11/main},
più precisamente dobbiamo modificare i file:
\begin{itemize}
\item \textbf{pg\_hba.conf}: Nello specifico va inserita un riga che specifica quali sono gli indirizzi IP di cui postgres può fidarsi.
\item \textbf{postgresql.conf}: Nello specifico va aggunta la riga \textbf{listen\_addresses = '*l}.
\end{itemize}
E' importante sottolineare che: \underline{!QUESTE IMPOSTAZIONI DEL SERVER SERVONO A FAR SI CHE}\\ 
\underline{L'ACCESSO E IL TESTING SIA PIU' VOLCE, NON E' PER NULLA SICURO USARE QUESTE}\\
\underline{IMPOSTAZIONI IN UN PRODOTTO FINALE!}

\end{document}