\documentclass[legalpaper]{article}
\usepackage[legalpaper, margin=1in]{geometry}
\usepackage[T1]{fontenc}
\usepackage[utf8]{inputenc}
\usepackage[italian]{babel}
\begin{document}

\section{Definizione specifiche}
Per scrivere il nostro schema e per descrivere le specifiche e realizzare la relazione abbiamo utilizzato questa convenzione:
\begin{itemize}
\item Entità: prima lettera MAIUSCOLA e il resto minuscolo\\
Esempio: Assistenza, Cliente, Intervento.
\item Relazione: tutta la parola minuscola\\
Esempio: è capace di risolvere, composta da.
\item Attibuti delle entità e delle relazioni: prima lettera minuscola e il resto della parola cammellizzato\\
Esempio: dataFineAssistenza, numeroIntervento, dataAssunzione.
\item Rappresentazione di un attributo di un entità o di una relazione: utilizzato il simbolo      ->      tra l'entità e l'attributo\\
Esempio: Entità->attributoUno, Entità->attributoDue, Intervento->modalità.
\end{itemize}

\section{Progettazione concettuale}

\subsection{Vincoli d'integrità}
I vincoli d'integrità sono tutte le proprietà di una base-di-dati, sono esprimibili 
tramite dei predicati; questi devono essere veri per garantire la validità dello schema.\\
\newline
Di seguito sono riportati i vincoli d'integrità presenti nel modello E-R:
\begin{itemize}
    \item L'attributo Assistenza->dataFineAssistenza deve essere maggiore o uguale\\ della proprietà richiede->data.
    \item Se Assistenza->dataFineAssistenza non è NULL allora: il numero di Interventi appartenenti a un'Assistenza deve essere minore o uguale alla differenza (in giorni) tra \\richiede->data e Assistenza->dataFineAssistenza.\\Questo vincolo deriva dal fatto che è possibile fare un Intervento al giorno per singola Assistenza.
   \item L'attributo Intervento->durata deve essere un valore nel dominio (0, 24).
Il valore della durata è rappresentato tramite il sistema orario a 24 ore.
  \item Un Tecnico non deve essere assegnato a degli Interventi diversi nello stesso giorno.
 
\end{itemize}

\end{document}
