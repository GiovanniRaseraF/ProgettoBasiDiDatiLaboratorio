\documentclass[legalpaper]{article}
\usepackage[legalpaper, margin=1in]{geometry}
\usepackage[T1]{fontenc}
\usepackage[utf8]{inputenc}
\usepackage[italian]{babel}
\begin{document}

\section{Progettazione concettuale}

\subsection{Vincoli d'integrità}
I vincoli d'integrità sono tutte le proprietà di una base-di-dati, sono esprimibili 
tramite dei predicati; questi devono essere veri per garantire la validità dello schema.\\
\newline
Di seguito sono riportati i vincoli d'integrità presenti nel modello E-R:
\begin{itemize}
    \item L'attributo Assistenza->dataFineAssistenza deve essere maggiore o uguale\\ della proprietà richiede->data.
    \item Se Assistenza->dataFineAssistenza non è NULL allora: il numero di Interventi appartenenti a un'Assistenza deve essere minore o uguale alla differenza (in giorni) tra \\richiede->data e Assistenza->dataFineAssistenza.\\Questo vincolo deriva dal fatto che è possibile fare un Intervento al giorno per singola Assistenza.
 
\end{itemize}
 
\end{document}
