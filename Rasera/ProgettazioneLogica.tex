\documentclass[legalpaper]{article}
\usepackage[legalpaper, margin=1in]{geometry}
\usepackage[T1]{fontenc}
\usepackage[utf8]{inputenc}
\usepackage[italian]{babel}
\begin{document}

\section{Progettazione logica}

\subsection{Regole di derivazione}
\textbf{Tecnico->oreLavorateMensilmente:} attributo derivato che contiene la somma delle ore che un Tecnico ha lavorato nel mese corrente ed è un valore che deve essere maggiore o uguale a zero.\\
Vale sempre che: 
\begin{itemize}
	\item  0 <= \textbf{Tecnico->oreLavorateMensilmente}
		\begin{itemize}\item sarà zero nel momento in cui il Tecnico non ha ancora lavorate nel mese corrente\end{itemize}
	\item Somma(\textbf{Intervento->durata} inerente al tecnico interessato) = \textbf{Tecnico->oreLavorateMensilmente}
		\begin{itemize}
		\item verrà aggiornato ogni volta che viene aggiunto un Intervento a qui un Tecnico lavorerà
		\end{itemize}
\end{itemize}
\subsection{Vincoli d'integrità}
	\begin{enumerate}
	\item Nel ciclo tra [Richiesta d'Assistenza, Intervento, Tecnico, Tipologia Guasto]
		\begin{itemize}
		\item Un Tecnico non deve essere assegnato ad un Intervento che fa riferimento ad una Richiesta d'Assistenza inerente ad una Tipologia Guasto che quel determinato tecnico non sa risolvere
		\end{itemize}
	\item Dopo che ho asseganto un valore all'attibuto \textbf{Richiesta d'Assistenza->dataFineAssistenza} non possono esistere degli Interventi inerenti a quella Richiesta che hanno l'attributo \textbf{Intervento->data} maggiore di \textbf{Richiesta d'Assistenza->dataFineAssistenza}
\end{enumerate}
 
\end{document}