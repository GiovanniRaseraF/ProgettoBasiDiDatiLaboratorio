\documentclass[legalpaper]{article}
\usepackage[legalpaper, margin=1in]{geometry}
\usepackage[T1]{fontenc}
\usepackage[utf8]{inputenc}
\usepackage[italian]{babel}
\usepackage{graphicx}

\begin{document}
	\subsection{Analisi delle ridondanze ---> 4.4!!!!}
	Nello schema ER è presente una singola ridondanza nell'entità Tecnico. Le ridondanze risultano essere utili perché possono essere informazioni più efficienti da ricavare direttamente invece che utilizzare apposite query per la loro determinazione. Essendo quindi derivabili, bisogna tenere aggiornata questa informazione.\\
	Per tenere una ridondanza presente all'interno dello schema, dobbiamo analizzare tutti i costi relativi ad esso e confrontarli con quelli che si ottengono qualora si ottenesse quella informazione in modo derivato.
	
	\subsubsection{Tabella dei volumi}
	Analizziamo tutte le componenti che si interfacciano con la ridondanza: \\ \newline
	\medskip
	\renewcommand\arraystretch{1,5}
	\begin{tabular}{|p{4cm}|p{4cm}|p{4cm}|}
		\hline
		\multicolumn{3}{|c|}{\textbf{Elementi coinvolti}}\\
		\hline
		Tecnico & \centering{E} & 75\\
		\hline
		gestito da & \centering{R} & 600.000\\
		\hline
		Intervento & \centering{E} & 600.000\\
		\hline
	\end{tabular}
	
	\subsubsection{Tabella delle operazioni}	
	Analizziamo tutte le operazioni che agiscono direttamente con la ridondanza: \\ \newline
	\medskip
	\renewcommand\arraystretch{1,5}
	\begin{tabular}{|p{4cm}|p{4cm}|p{4cm}|}
		\hline
		\multicolumn{3}{|c|}{\textbf{Operazioni coinvolte}}\\
		Operazione 3 & I & 450 a settimana\\
		\hline
		Operazione 6 & \centering{B} & 1 al giorno\\
		\hline
	\end{tabular}

	\subsubsection{Tabella degli accessi in riferimento all'operazione 3}
	Il costo di una operazione (calcolato usando una tabella dei volumi) e lo schema di navigazione possono essere riassunti nella tabella degli accessi: \\ \newline
	\begin{tabular}{|p{4cm}|p{4cm}|p{4cm}|p{4cm}|}
		\hline
		\multicolumn{4}{|c|}{\textbf{Tabella degli accessi senza la presenza di ridondanza}}\\
		\hline
		Concept & Type & Accesses & Type \\
		\hline
		Tecnico & \centering{E} & 1 & R \\
		\hline
		gestito da & R & 1 & R \\ 
		\hline
		Intervento & E & 1 & W \\
		\hline
	\end{tabular} \\
	Analizziamo ora i costi:
	\begin{itemize}
		\item 75 * 2 = 150 (costi in riferimento all'accesso a Tecnico e gestito da)
		\item 75 * 2 = 150 (costi in riferimento alla scrittura in Intervento)
		\item 150 + 150 = 300 (costo totale giornaliero dell'operazione)
	\end{itemize}
	\begin{tabular}{|p{4cm}|p{4cm}|p{4cm}|p{4cm}|}
		\hline
		\multicolumn{4}{|c|}{\textbf{Tabella degli accessi con la presenza di ridondanza}}\\
		\hline
		Concept & Type & Accesses & Type \\
		\hline
		Tecnico & \centering{E} & 1 & W \\
		\hline
		gestito da & R & 1 & R \\ 
		\hline
		Intervento & E & 1 & W \\
		\hline
	\end{tabular} \\
	Analizziamo ora i costi:
	\begin{itemize}
		\item 75 * 2 = 150 (costi in riferimento ala modifica di Tecnico)
		\item 75 (costi in riferimento alla relazione gestito da)
		\item 75 * 2 = 150 (costi in riferimento alla scrittura in Intervento)
		\item 150 + 75 +150 = 375 (costo totale giornaliero dell'operazione)
	\end{itemize}
	
	Come possiamo notare, il costo con la presenza di ridondanza è leggermente maggiore (esattamente del 25\%), ma come vedremo nella prossima operazione, questo ci comporterà un beneficio.
	
	\subsubsection{Tabella degli accessi in riferimento all'operazione 6}
	Il costo di una operazione (calcolato usando una tabella dei volumi) e lo schema di navigazione possono essere riassunti nella tabella degli accessi: \\ \newline
	\medskip
	\renewcommand\arraystretch{1,5}
	\begin{tabular}{|p{4cm}|p{4cm}|p{4cm}|p{4cm}|}
		\hline
		\multicolumn{4}{|c|}{\textbf{Tabella degli accessi in presenza di ridondanza}}\\
		\hline
		Concept & Type & Accesses & Type \\
		\hline
		Tecnico & \centering{E} & 1 & R \\
		\hline
	\end{tabular}\\
	\newline
	Tenendo in considerazione che il costo di una operazione di read è uguale ad un accesso, con la presenza della ridondanza, ottenere l'informazione derivabile (Tecnico->oreLavorateMensilmente) ci viene a costare esattamente 75 accessi al giorno cioè:
	\begin{itemize}
		\item 75 * 1 = 75 accessi giornalieri
		\item 75 * 26 = 1950 accessi totali mensili
	\end{itemize}
	\begin{tabular}{|p{4cm}|p{4cm}|p{4cm}|p{4cm}|}
		\hline
		\multicolumn{4}{|c|}{\textbf{Tabella degli accessi senza la presenza di ridondanza}}\\
		\hline
		Concept & Type & Accesses & Type \\
		\hline
		Tecnico & \centering{E} & 1 & R \\
		\hline
		gestito da & R & 1 & R \\ 
		\hline
		Intervento & E & 1 & R \\
		\hline
	\end{tabular} \\
	\newline
	Senza la presenza della ridondanza, è necessario:
	\begin{itemize}
		\item 75 * 3 = 225 accessi al giorno
		\item 225 * 26 = 5850 accessi mensili
	\end{itemize}
	
	Conclusioni: vista l'analisi svolta, è facile capire come la presenza della ridondanza risulti conveniente in quanto il numero di accessi è inferiore del 67\%.\\
	Inoltre, c'è da tenere in considerazione che ogni volta che si effettua l'accesso alla tabella degli Interventi, bisogna filtrare 600.000 righe.
	



	
	
	
	
	
	
			
\end{document}