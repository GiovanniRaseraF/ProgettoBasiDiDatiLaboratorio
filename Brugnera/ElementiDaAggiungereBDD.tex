\documentclass[legalpaper]{article}
\usepackage[legalpaper, margin=1in]{geometry}
\usepackage[T1]{fontenc}
\usepackage[utf8]{inputenc}
\usepackage[italian]{babel}
\usepackage{graphicx}

\begin{document}
	\section{Progettazione logica}
		Lo scopo della progettazione logica è di costruire uno schema relazionale che rappresenti in modo accurato, efficiente e soprattutto correttamente tutte le informazioni descritte da uno schema ER prodotto durante la fase precedente. \\
		Questo non è una semplice trasformazione da un modello ad un altro per due motivi:
		\begin{itemize}
			\item non tutti i costrutti del modello ER possono essere tradotti nel modello relazionale;
			\item lo schema deve essere ristrutturato in modo che l'esecuzione delle operazioni avvenga il più efficientemente possibile
		\end{itemize}
		Inoltre si controllano e governano le ridondanze. Infatti per analizzarle si usano: 
		\begin{itemize}
			\item i volumi dei dati;
			\item operazioni attese;
			\item frequenza delle operazioni;
		\end{itemize}
		\'E utile dividere questo tipo di progettazione in due semplici step:
		\begin{itemize}
			\item ristrutturazione dello schema ER, basato sull'ottimizzazione e semplificazione dello schema;
			\item traduzione nel modello logico.
		\end{itemize}
		\begin{figure}[ht]
			\includegraphics[width=4cm]{Schema Prog. Logica}
		\end{figure}
	\subsection{Tabella dei volumi}
		In questa sezione andiamo a definire il numero di occorrenze per ogni entità e relazione presente all'interno dello schema ER. Viene ipotizzato che i volumi facciano riferimento all'attività dopo i suoi dieci anni di vita.\\
		\newline
		\renewcommand\arraystretch{2}
		\begin{tabular}{ |p{5cm}|p{2cm}|p{5cm}| }
			\hline
			\multicolumn{3}{|c|}{\textbf{Tabella dei volumi}} \\
			\hline
			\textbf{Entità/Relazione} & \textbf{Tipo} & \textbf{Volume} \\
			\hline
			Azienda & E & 700 \\ \hline
			Ente Pubblico & E & 300 \\ \hline
			Persona Giuridica & E & 1000 \\ \hline
			Singolo Cittadino & E & 1000 \\ \hline
			Cliente & E & 2000 \\ \hline
			effettua & R & 40000 \\ \hline
			Richiesta & E & 40000 \\ \hline
			per & R & 40000 \\ \hline
			Assistenza & E & 40000 \\ \hline
			inerente & R & 40000 \\ \hline
			Guasto & E & 40000 \\ \hline
			composto da & R & 19000 \\ \hline
			Intervento & E & 19000 \\ \hline
			gestito da & R & 110 \\ \hline
			è capace di risolvere & R & 40000 \\ \hline
			Tecnico & E & 110 \\ \hline
		\end{tabular}

	
	\subsection{Tabella delle operazioni}
	Per ogni operazione indicata precedentemente, andiamo a definire la frequenza con la quale essa viene eseguita e la sua tipologia:
	\begin{itemize}
		\item \textbf{Batch}: operazioni che si possono "ignorare", ovvero vengono svolte quando il sistema non lavora in pieno regime (ad esempio tarda sera). Facendo così, si lascia spazio alle operazioni più importanti;
		\item \textbf{Interactive}: operazioni più importanti, dove la velocità di esecuzione deve essere veloce. Il tempo di risposta quindi deve essere veloce.
	\end{itemize}
		\renewcommand\arraystretch{2}
		\begin{tabular}{ |p{5cm}|p{2cm}|p{5cm}| }
			\hline
			\multicolumn{3}{|c|}{\textbf{Tabella delle operazioni}} \\
			\hline
			\textbf{Operazione} & \textbf{Tipo} & \textbf{Frequenza} \\
			\hline
			Operazione 1 & I &  media di 2 volte a settimana \\ \hline
			Operazione 2 & I & 40 volte a settimana \\ \hline
			Operazione 3 & I & 120 a settimana \\ \hline
			Operazione 4 & B & 6 volte all'anno \\ \hline
			Operazione 5 & I & 20 volte a settimana \\ \hline
			Operazione 6 & B & 1 volta a settimana \\ \hline
			Operazione 7 & B & 1 volta al mese \\ \hline
			Operazione 8 & I & 10 volte al giorno \\ \hline
			Operazione 9 & B & 1 volta al mese \\ \hline
			Operazione 10 & B & 1 volta a settimana \\ \hline
		
		\end{tabular}	 
	
	Cambiamenti operazioni e parametri vari dopo 10 anni di attività: 
	- numero clienti totale 10.000 (fedeli e non)
	- numero tecnici totali 75
	- media clienti annuali 1000
	- media di 2 richieste di assistenza per cliente all'anno
	- media di 3 interventi per assistenza
	- in media 300 clienti rimangono fedeli al nostro servizio (per fedeli intendo che faranno riferimento a noi per qualsiasi guasto negli anni a venire)
	
	Supponiamo che dei 10.000 clienti avuti in 10 anni, 3000 siano rimasti "fedeli" a noi (per fedeli intendo clienti che fanno richieste anche negli anni successivi). Quindi nell'undicesimo anno avremmo 3000+1000=4000 nuovi clienti (fedeli + eventuali fedeli o meno).
	Facendo così, avremo 4000*2 assistenze (perchè in media sono 2) e quindi ciò implica 24.000 interventi.
	Questi vengono divisi tra i 75 tecnici, facendo si che ognuno di essi abbia in media 320 interventi in un anno. Sono un pelo sfruttati ma non me ne frega un cazzo.
	nel dodicesimo anno abbiamo in totale 3300 clienti fedeli. arriviamo a fine anno con 4300 clienti (fedeli e non), con quindi 25.800 interventi fatti, ovvero 344 a tecnico. Risulta essere troppo oppressivo. Dobbiamo quindi aumentare il numero di tecnici da assumere con l'aumento della clientela, aumentando così il volume dell'azienda stessa. Se supponiamo che ogni anno vengono assunti 6 nuovi operai, abbiamo che il rapporto tecnici/interventi rimane sempre interno ai 320 (interventi per operaio).
	Ad esempio il tredicesimo anno avremmo 87 operai con 27.600 interventi, cioè intorno ai 317 interventi per operaio.
	

	Appunti vari: del 100 percento dei clienti, il 50 percento è ente pubblico e il resto è privato. Del 50 percento pubblico, il suo 70 percento fa riferimento alle aziende mentre il 30 percento agli enti pubblici.
	
	
	
	
	
			
\end{document}