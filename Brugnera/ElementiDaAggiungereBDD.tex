\documentclass[legalpaper]{article}
\usepackage[legalpaper, margin=1in]{geometry}
\usepackage[T1]{fontenc}
\usepackage[utf8]{inputenc}
\usepackage[italian]{babel}
\usepackage{graphicx}

\begin{document}
	\subsection{Analisi delle ridondanze ---> 4.4!!!!}
	Nello schema ER è presente una singola ridondanza nell'entità Tecnico. Le ridondanze risultano essere utili perché possono essere informazioni più efficienti da ricavare direttamente invece che utilizzare apposite query per la loro determinazione. Essendo quindi derivabili, bisogna tenere aggiornata questa informazione.\\
	Per tenere una ridondanza presente all'interno dello schema, dobbiamo analizzare tutti i costi relativi ad esso e confrontarli con quelli che si ottengono qualora si ottenesse quella informazione in modo derivato.
	
	\subsubsection{Tabella dei volumi}
	Analizziamo tutte le componenti che si interfacciano con la ridondanza: \\ \newline
	\medskip
	\renewcommand\arraystretch{1,5}
	\begin{tabular}{|p{4cm}|p{4cm}|p{4cm}|}
		\hline
		\multicolumn{3}{|c|}{\textbf{Elementi coinvolti}}\\
		\hline
		Tecnico & \centering{E} & .\\
		\hline
		gestito da & \centering{R} & .\\
		\hline
		Intervento & \centering{E} & .\\
		\hline
		
		
		
	\end{tabular}
			
\end{document}